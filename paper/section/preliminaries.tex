\section{Preliminaries: Programs and Probability Spaces}
\label{sec:prelims}

To formally define the model of \thelogic\ and validate its rules,
we introduce a number of preliminary notions.
Our starting point is the measure-theoretic approach of
\cite{lilac} in defining probabilistic separation.
We recall the main definitions here.
The main additional assumption we make throughout
is that the set of outcomes of distributions is countable.


\begin{definition}[Probability spaces]
\label{def:prob-sp}
  Given a set of possible \emph{outcomes} $\Outcomes$,
  a \salgebra\ $ \salg \in \SigAlg(\Outcomes) $ is
  a set of subsets of $\Outcomes$
    that is closed under countable unions and
    complements, and such that
    $\Outcomes \in \salg$.
  The \emph{full} \salgebra\ over~$\Outcomes$ is
  $ \Full{\Outcomes} = \powerset(\Outcomes) $,
		the powerset of $\Outcomes$.
  For $F \subs \powerset(\Outcomes)$, we write $\sigcl{F} \in \SigAlg(\Outcomes)$
    for the smallest \salgebra\ containing $F$.
Given~$\salg \in \SigAlg(\Outcomes)$,
  a \emph{probability distribution}~$\prob \in \Dist(\salg)$,
    is a countably additive function~$ \prob \from \salg \to [0,1] $
    with $\prob(\Outcomes)=1$.
  The support of a distribution $\prob \in \Dist(\Full{\Outcomes})$
  is the set of outcomes with non-zero probability
  $ \psupp(\prob) \is \set{ a \in \Outcomes | \prob(a) > 0 } $,
  where $\prob(a)$ abbreviates $\prob(\set{a})$.

  A \emph{probability space} $ \psp \in \ProbSp(\Outcomes) $ is
  a pair $ \psp = (\salg, \prob) $ of
    a \salgebra\ $\salg \in \SigAlg(\Outcomes)$ and
    a probability distribution~$\prob \in \Dist(\salg)$.
The \emph{trivial} probability space $\Triv{\Outcomes} \in \ProbSp(\Outcomes)$
  is the trivial \salgebra\ $ \set{\Outcomes,\emptyset} $
  equipped with the trivial probability distribution.
Given $\salg_1 \subs \salg_2$ and $\prob \in \Dist(\salg_2)$,
  the distribution $ \restr{\prob}{\salg_1} \in \Dist(\salg_1) $
  is the restriction of $\prob$ to $\salg_1$.
The \emph{extension pre-order}~$(\extTo)$ over probability spaces is defined as
  $
    (\salg_1, \prob_1) \extTo (\salg_2, \prob_2) \is
      \salg_1 \subs \salg_2 \land \prob_1 = \restr{\prob_2}{\salg_1}.
  $

  A function $f \from \Outcomes_1 \to \Outcomes_2$ is
  \em{measurable} on
  $ \salg_1\in\SigAlg(\Outcomes_1)$ and $\salg_2\in\SigAlg(\Outcomes_2) $
  if\/
  $ \forall \event \in \salg_2 \st {\inv{f}(\event) \in \salg_1} $.
When $\salg_2 = \Full{\Outcomes_2}$
	we simply say~$f$ is measurable~in~$\salg_1$.
\end{definition}

\begin{definition}[Product and union spaces]
\label{def:prod-union-sp}
  Given $ \salg_1 \in \SigAlg(\Outcomes_1),\salg_2 \in \SigAlg(\Outcomes_2) $,
  their product is the \salgebra{}
  $ \salg_1 \pprod \salg_2 \in \SigAlg(\Outcomes_1 \times \Outcomes_2) $
  defined as
  $ \salg_1 \pprod \salg_2 \is \sigcl{\set{\event_1 \times \event_2 | \event_1 \in \salg_1, \event_2 \in \salg_2}} $,
  and their union is the \salgebra{}
  $ \salg_1 \punion \salg_2 \is \sigcl{\salg_1 \union \salg_2} $.
The product of two probability distributions
  $ \prob_1 \in \Dist(\salg_1) $ and
  $ \prob_2 \in \Dist(\salg_2) $ is
  the distribution
  $ (\prob_1 \pprod \prob_2) \in \Dist(\salg_1 \pprod \salg_2) $
  defined by
  $ (\prob_1 \pprod \prob_2)(\event_1 \times \event_2) = \prob_1(\event_1)\prob_2(\event_2) $
  for all $\event_1 \in \salg_1$, $\event_2 \in \salg_2$.
\end{definition}

\begin{definition}[Independent product~\cite{lilac}]
\label{def:indep-comb}
  Given $ (\salg_1, \prob_1),(\salg_2, \prob_2) \in \ProbSp(\Outcomes) $,
  their \emph{independent product} is
  the probability space
  $(\salg_1 \punion \salg_2, \prob) \in \ProbSp(\Outcomes)$
  where for all~$ \event_1 \in \salg_1, \event_2 \in \salg_2 $,
  $
    \prob(\event_1 \inters \event_2) = \prob_1(\event_1) \cdot \prob_2(\event_2)
  $. It is unique, if it exists~\cite[Lemma 2.3]{lilac}. Let $ \psp_1 \iprod \psp_2 $ be the unique independent product
  of $\psp_1$ and $\psp_2$ when it exists, and be undefined otherwise.
\end{definition}





\paragraph{Indexed tuples}
To handle unary and higher-arity relational assertions in a uniform way,
we consider finite sets of indices $I \subs \Nat$,
and \pre I-indexed tuples of values of type~$X$,
represented as (finite) functions $\Hyp[I]{X}$.
We use boldface to range over such functions.
The syntax $ \m{x} = \m[i_0: x_0,\dots,i_n: x_n] $ denotes the function
$ \m{x} \in \Hyp[\set{i_0,\dots,i_n}]{X} $ with $\m{x}(i_k) = x_k$.
We often use comprehension-style notation \eg $\m{x} = \m[i: x_i | i\in I]$.
For $\m{x} \in \Hyp[I]{A}$ we let $\supp{\m{x}} \is I$.
Given some $ \m{x} \in \Hyp[I]{A} $ and some $J \subs I$,
the operation $ \m{x} \setminus J \is \m[i: \m{x}(i) | i \in I \setminus J] $
removes the components with indices in~$J$ from $\m{x}$.

\paragraph{Programs}
We consider a simple first-order imperative language.
We fix a finite set of \emph{program variables}~$\p{x} \in \Var$
and countable set of \emph{values}~$\val \in \Val \is \Int$
and define the program \emph{stores} to be
$ \store \in \Store \is \Var \to \Val $
(note that $\Store$ is countable).

\begin{mathfig}
  \begin{grammar}
\Expr \ni \expr \is
        \val | \code{x} | \prim(\vec{\expr})
    \qquad
    \vec{\expr} \grammIs \expr_1,\dots,\expr_n
    \qquad
    \prim \grammIs + | - | < | \dots
    \qquad
    \dist \grammIs \p{Ber} | \p{Unif} | \dots
    \\
    \Term \ni \term \is
        \Assn{x}{\expr}
      | \Sample{x}{\dist}{\vec{\expr}}
      | \Skip
| \Seq{\term_1}{\term_2}
      | \Cond{\expr}{\term_1}{\term_2}
      | \Loop{\expr}{\term}
  \end{grammar}
  \caption{Syntax of program terms.}
  \label{fig:term-syntax}
\end{mathfig}

Program \emph{terms}~$ \term \in \Term $ are formed according to the
grammar in \cref{fig:term-syntax}.
For simplicity,
booleans are encoded by using~$0 \in \Val$ as false and any other value as true.
We will use the events
  $\false \is \set{0}$ and
  $\true \is \set{n \in \Val | n\ne 0}$.
Programs use standard deterministic primitives~$\prim$,
which are interpreted as functions
$ \sem{\prim} \from \Val^n \to \Val $, where $n$ is the arity of $\prim$.
Expressions~$\expr$ are effect-free deterministic numeric expressions,
and denote, as is standard, a function
$ \sem{\expr} \from \Store \to \Val $,
\ie a random variable of $\Full{\Store}$.
We write $\pvar(\expr)$ for the set of program variables that occur
in~$\expr$.
Programs can refer to some collection of known
\emph{discrete} distributions~$\dist$,
each allowing a certain number of parameters.
Sampling assignments $ \code{x:~$\dist$($\vec{v}$)} $
sample from the distribution $\Sem{\dist}(\vec{v}) \from \Dist(\Full{\Val})$.
The distribution $ \Sem{\p{Ber}}(p) = \Ber{p} \of\Dist(\Full{\set{0,1}}) $
is the Bernoulli distribution assigning probability~$p$ to outcome~1.

Similar to Lilac, we consider a simple iteration construct
$ \code{repeat}\; e\; t $ which evaluates $e$ to a value $n \in \Val$
and, if $n>0$, executes $t$ in sequence $n$ times.
This means we only consider a subset of terminating programs.
The semantics of programs is
entirely standard and is defined in \appendixref{sec:appendix:definition}.
It associates each term~$t$ to a function
$
  \sem{t} \from \Dist(\Full{\Store}) \to \Dist(\Full{\Store})
$
from distributions of input stores to
distributions of output stores.

In the relational reasoning setting, one would consider multiple
programs at the same time and relate their semantics.
Following LHC~\cite{d2022proving},
we define \emph{hyper-terms} as $ \m{t} \in \Hyp[J]{\Term} $
for some finite set of indices~$J$.
Let~$I$ be such that~$\supp{\m{t}} \subs I$; the semantics
$
  \sem{\m{t}}_I \from
      \Hyp[I]{\Dist(\Full{\Store})} \to \Hyp[I]{\Dist(\Full{\Store})}
$
takes a \pre I-indexed family of distributions as input and outputs
another \pre I-indexed family of distributions:
\[
  \sem{\m{t}}_I(\m{\prob}) \is
    \fun i.
    \ITE{i \in \supp{\m{t}}}{
      \sem{\m{t}(i)}(\m{\prob}(i))
    }{
      \m{\prob}(i)
    }
\]
Note that the store distributions at indices in $ I \setminus \supp{t} $
are preserved as is.
We omit~$I$ when it can be inferred from context.
To refer to program variables in a specific component we use
elements of $I\times \Var$, writing $\ip{x}{i}$ for $(i,\p{x})$.

