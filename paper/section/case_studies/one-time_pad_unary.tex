\subsection{One-time Pad (Unary)}
\label{sec:appendix:ex:otp-unary}

  Similarly to the relational version, we can, using \ref{rule:wp-seq},\ref{rule:wp-assign} and \ref{rule:wp-samp}, easily show that:
\begin{equation*}
  \True\withp{\m{\permap}}
  \proves
  \WP {
   \m[\I1: \code{encrypt}()]
  }{
     \distAs{\Ip{k}{1}}{\Ber{\onehalf}} *
     \distAs{\Ip{m}{1}}{\Ber{p}} *
     \sure{\Ip{c}{1} = \Ip{k}{1} \xor \Ip{m}{1}}
  }.
\end{equation*}

We then show the crucial derivation of \cref{sec:ex:one-time-pad} in more detail.
\begin{eqexplain}
  &
  \distAs{\Ip{k}{1}}{\Ber{\onehalf}} *
  \distAs{\Ip{m}{1}}{\Ber{p}} *
  \sure{\Ip{c}{1} = \Ip{k}{1} \xor \Ip{m}{1}}
\whichproves
  \CC{\Ber{p}} m.
  \bigl(
    \sure{\Ip{m}{1}=m} *
    \distAs{\Ip{k}{1}}{\Ber{\onehalf}} *
    \sure{\Ip{c}{1} = \Ip{k}{1} \xor \Ip{m}{1}}
  \bigr)
  \byrules{c-unit-r,c-frame}
\whichproves
  \CC{\Ber{p}} m.
  \bigl(
    \sure{\Ip{m}{1}=m} *
    \distAs{\Ip{k}{1}}{\Ber{\onehalf}} *
    \sure{\Ip{c}{1} = \Ip{k}{1} \xor m}
  \bigr)
  \byrules{sure-merge}
\whichproves
  \CC{\Ber{p}} m.
  \Bigl(
    \sure{\Ip{m}{1}=m} *
    \CC{\Ber{\onehalf}} k.
    \bigl(
      \sure{\Ip{k}{1}=k} *
      \sure{\Ip{c}{1} = \Ip{k}{1} \xor m}
    \bigr)
  \Bigr)
  \byrules{c-unit-r,c-frame}
\whichproves
  \CC{\Ber{p}} m.
    \bigl(
      \sure{\Ip{m}{1}=m} *
      \CC{\Ber{\onehalf}} k.
        \sure{\Ip{k}{1}=k \land \Ip{c}{1} = k \xor m}
    \bigr)
  \byrules{sure-merge}
\whichproves
  \CC{\Ber{p}} m.
    \left(
    \sure{\Ip{m}{1}=m} *
    \begin{cases}
      \CC{\Ber{\onehalf}} k. \sure{\Ip{c}{1}=k} \CASE m=0
      \\
      \CC{\Ber{\onehalf}} k. \sure{\Ip{c}{1}=\neg k} \CASE m=1
    \end{cases}
    \right)
  \byrule{c-cons}
\whichproves
  \CC{\Ber{p}} m.
    \left(
    \sure{\Ip{m}{1}=m} *
    \begin{cases}
      \CC{\Ber{\onehalf}} k. \sure{\Ip{c}{1}=k} \CASE m=0
      \\
      \CC{\Ber{\onehalf}} k. \sure{\Ip{c}{1}=k} \CASE m=1
    \end{cases}
    \right)
  \byrule{c-transf}
\whichproves
  \CC{\Ber{p}} m.
    \bigl(
      \sure{\Ip{m}{1}=m} *
      \CC{\Ber{\onehalf}} k. \sure{\Ip{c}{1}=k}
    \bigr)
\whichproves
  \CC{\Ber{p}} m.
  \CC{\Ber{\onehalf}} k.
    \bigl(
      \sure{\Ip{m}{1}=m} *
      \sure{\Ip{c}{1}=k}
    \bigr)
  \byrules{c-frame}
\whichproves
  \CC{\Ber{p}} m.
  \CC{\Ber{\onehalf}} k.
    \sure{\Ip{m}{1}=m \land \Ip{c}{1}=k}
  \byrules{sure-merge}
\whichproves
  \CC{\Ber{p} \pprod \Ber{\onehalf}} (m,k).
    \sure{(\Ip{m}{1},\Ip{c}{1})=(m,k)}
  \byrules{c-assoc}
\whichproves
  \distAs{(\Ip{m}{1},\Ip{c}{1})}{(\Ber{p} \pprod \Ber{\onehalf})}
  \byrule{c-unit-r}
\whichproves
  \distAs{\Ip{m}{1}}{\Ber{p}} *
  \distAs{\Ip{c}{1}}{\Ber{\onehalf}}
  \byrule{prod-split}
\end{eqexplain}

The application of \ref{rule:c-transf} to the case with $m=1$ is as follows:
\[
\infer{
  \forall b \in \set{0,1}.
    \Ber{\onehalf}(b)=\Ber{\onehalf}(\neg b)
}{
  \CC{\Ber{\onehalf}} k. \sure{\Ip{c}{1}=\neg k}
  \proves
  \CC{\Ber{\onehalf}} k. \sure{\Ip{c}{1}=k}
}
\] 
\begin{figure*}
  \adjustfigure[\small]\setlength\tabcolsep{0pt}\begin{tabular*}{\textwidth}{
    @{\extracolsep{\fill}}
    *{4}{p{\textwidth/4}}@{}
  }
\begin{sourcecode*}
def BelowMax($x$,$S$):
  repeat $N$:
    q:~$\prob_S$
    r':=r
    r := r' || q >= $x$
\end{sourcecode*}
&
\begin{sourcecode*}
def AboveMin($x$,$S$):
  repeat $N$:
    p:~$\prob_S$
    l':=l
    l := l' || p <= $x$
\end{sourcecode*}
&
\begin{sourcecode*}
def BETW_SEQ($x$, $S$):
  BelowMax($x$,$S$);
  AboveMin($x$,$S$);
  d := r && l
\end{sourcecode*}
\\
\begin{sourcecode*}
def BETW($x$,$S$):
  repeat $2 N$:
    s:~$\prob_S$
    l':=l
    l := l' || s <= $x$
    r':=r
    r := r' || s >= $x$
  d := r && l
\end{sourcecode*}
&
\begin{sourcecode*}
def BETW_MIX($x$, $S$):
  repeat $N$:
    p:~$\prob_S$
    l':=l
    l := l' || p <= $x$
    q:~$\prob_S$
    r':=r
    r := r' || q >= $x$
  d := r && l
\end{sourcecode*}
&
\begin{sourcecode*}
def BETW_N($x$,$S$):
  repeat $N$:
    s:~$\prob_S$
    l':=l
    l := l' || s <= $x$
    r':=r
    r := r' || s >= $x$
  d := r && l
\end{sourcecode*}
\end{tabular*}   \caption{Stochastic dominance examples: composing Monte Carlo algorithms in different ways. All variables are initially 0.}
  \label{fig:between-code-repeat}
\end{figure*}
