\subsection{Monte Carlo: Equivalence Between \p{BETW\_MIX} and \p{BETW\_SEQ}}

  \Cref{fig:between-code-repeat} shows another way in which one can
approximately compute the ``between'' function: \p{BETW\_MIX}.
In this example we want to prove the equivalence between
\p{BETW\_MIX} and \p{BETW\_SEQ}.
Again, the main obstacle to overcome in the proof is that
the structure of the two programs is very different.
\p{BETW\_SEQ} has two loops of~$N$ iterations, with one sample per iteration.
\p{BETW\_MIX} has a single loop of~$N$ iterations, but it samples twice
per iteration.
Note that the equivalence cannot be understood as a generic program transformation: the order in which the samples are taken in the two programs
is drastically different; they are only equivalent because the calculations
done on each of these independent samples are independent from one another.

Intuitively, we want to produce a proof that aligns each iteration
of the first loop of \p{BETW\_SEQ} with half of each iteration of \p{BETW\_MIX},
and each iteration of the second loop of \p{BETW\_SEQ} with the second half of each iteration of \p{BETW\_MIX}.
In the same vein as the previous example, we want to formalize the
proof pattern as a rule that aligns the loops as desired,
prove the rule is derivable, and apply it to the example.
A rule encoding the above pattern is the following:
\begin{proofrule}
  \infer*[lab=wp-loop-mix]{
  \forall i < N \st
    P_1(i) \proves \WP{\m[\I1: t_1, \I2: t_1']}{P_1(i+1)}
  \\
  \forall i < N \st
    P_2(i) \proves \WP{\m[\I1: t_2, \I2: t_2']}{P_2(i+1)}
}{
  P_1(0) * P_2(0)
  \proves
  \WP{\m[
    \I1: (\Loop{N}{t_1}\p;\Loop{N}{t_2}),
    \I2: \Loop{N}{(t_1';t_2')}
  ]}{P_1(N) * P_2(N)}
}   \relabel{rule:wp-loop-mix}
\end{proofrule}
Before showing how this rule is derivable,
which we do in \cref{proof:wp-loop-mix},
let us show how to use it to close our example.

We want to prove the goal:
\[
  \begin{conj}
    \sure{\Ip{r}{1} = \Ip{l}{1} = 0} \land
    \sure{\Ip{r}{2} = \Ip{l}{2} = 0}
  \end{conj}
  \withp{\m{p}}
  \proves
  \WP{\m<
    \I1: \Loop{N}{t_{\p{r}}}\p; \Loop{N}{t_{\p{l}}},
    \I2: \Loop{N}{(t_{\p{r}}\p; t_{\p{l}})}
  >}*{
    \cpl*{
    \begin{conj*}
      \Ip{r}{1} = \Ip{r}{2} \land
      \Ip{l}{1} = \Ip{l}{2}
    \end{conj*}
    }
  }
\]
where
  $\m{p}$ has full permissions for all the relevant variables,
  $t_{\p{r}}$ is the body of the loop of \p{BelowMax}, and
  $t_{\p{l}}$ is the body of the loop of \p{AboveMin}.

As a first manipulation, we use \ref{rule:rl-merge} in the postcondition,
and \cref{rule:coupling} (via \ref{rule:sure-dirac}) to the precondition,
to obtain:
\[
  \begin{conj}
    \cpl{\Ip{r}{1} = \Ip{r}{2}}\withp{\m{p}_{\p{r}}} * {}\\
    \cpl{\Ip{l}{1} = \Ip{l}{2}}\withp{\m{p}_{\p{l}}}
  \end{conj}
  \proves
  \WP{\m<
    \I1: \Loop{N}{t_{\p{r}}}\p; \Loop{N}{t_{\p{l}}},
    \I2: \Loop{N}{(t_{\p{r}}\p; t_{\p{l}})}
  >}*{
    \begin{conj}
      \cpl{\Ip{r}{1} = \Ip{r}{2}}\withp{\m{p}_{\p{r}}} * {}\\
      \cpl{\Ip{l}{1} = \Ip{l}{2}}\withp{\m{p}_{\p{l}}}
    \end{conj}
  }
\]
where
$\m{p}_{\p{r}} = \m[\Ip{r}{1}:1, \Ip{r}{2}:1, \Ip{q}{1}:1, \Ip{q}{2}:1] $, and
$\m{p}_{\p{l}} = \m[\Ip{l}{1}:1, \Ip{l}{2}:1, \Ip{p}{1}:1, \Ip{p}{2}:1] $.
Then \ref{rule:wp-loop-mix} applies and we are left with the two triples
\begin{gather*}
  \cpl{\Ip{r}{1} = \Ip{r}{2}}
  \withp{\m{p}_{\p{r}}}
  \proves
  \WP{\m[
    \I1: t_{\p{r}},
    \I2: t_{\p{r}}
  ]}*{
    \cpl{\Ip{r}{1} = \Ip{r}{2}}
    \withp{\m{p}_{\p{r}}}
  }
  \\
  \cpl{\Ip{l}{1} = \Ip{l}{2}}
  \withp{\m{p}_{\p{l}}}
  \proves
  \WP{\m[
    \I1: t_{\p{l}},
    \I2: t_{\p{l}}
  ]}*{
    \cpl{\Ip{l}{1} = \Ip{l}{2}}
    \withp{\m{p}_{\p{l}}}
  }
\end{gather*}
which are trivially proved using a standard coupling argument.

\medskip
As promised, we now prove \ref{rule:wp-loop-mix} is derivable,
concluding the example.
\begin{lemma}
\label{proof:wp-loop-mix}
  \Cref{rule:wp-loop-mix} is sound.
\end{lemma}

\begin{proof}
  Assume:
  \begin{gather}
    \forall i < N \st
      P_1(i) \proves \WP{\m[\I1: t_1, \I2: t_1']}{P_1(i+1)}
    \label{wp-loop-mix:P1}
    \\
    \forall i < N \st
      P_2(i) \proves \WP{\m[\I1: t_2, \I2: t_2']}{P_2(i+1)}
    \label{wp-loop-mix:P2}
  \end{gather}
  Our goal is to prove:
  \[
    P_1(0) * P_2(0)
    \proves
    \WP{\m[
      \I1: (\Loop{N}{t_1}\p;\Loop{N}{t_2}),
      \I2: \Loop{N}{(t_1';t_2')}
    ]}{P_1(N) * P_2(N)}
  \]
  We first massage the goal to split the sequential composition at \I1.
  By \ref{rule:wp-seq} and \ref{rule:wp-nest} we obtain
  \[
    P_1(0) * P_2(0)
    \proves
    \WPv{\m[\I2: \Loop{N}{(t_1';t_2')}]}{
      \WP{\m[\I1: \Loop{N}{t_1}]}{
        \WP{\m[\I1: \Loop{N}{t_2}]}{
          P_1(N) * P_2(N)
        }
      }
    }
  \]
  Now by applying \ref{rule:wp-frame} in the postcondition (twice) we obtain
  \begin{equation}
    P_1(0) * P_2(0)
    \proves
    \WP{\m[\I2: \Loop{N}{(t_1';t_2')}]}*{
      \begin{conj}
      \WP{\m[\I1: \Loop{N}{t_1}]}{P_1(N)} * {}\\
      \WP{\m[\I1: \Loop{N}{t_2}]}{P_2(N)}
      \end{conj}
    }
  \label{wp-loop-mix:goal-loop}
  \end{equation}
  Define
  \begin{align*}
    P(i) &\is Q_1(i) * Q_2(i)
    &
    Q_1(i) &\is \WP{\m[\I1: \Loop{i}{t_1}]}{P_1(i)}
    &
    Q_2(i) &\is \WP{\m[\I1: \Loop{i}{t_2}]}{P_2(i)}
  \end{align*}
  Clearly we have
  $ P_1(0) * P_2(0) \proves P(0) $ (by \ref{rule:wp-loop-0})
  and $ P(N) $ coincides with the postcondition
  of our goal~\eqref{wp-loop-mix:goal-loop}, which is now rewritten to:
  \[
    P(0)\proves \WP{\m[\I2: \Loop{N}{(t_1'\p;t_2')}]}{P(N)}
  \]
  Now we can apply \ref{rule:wp-loop} with invariant~$P$ and reduce the goal to
  the triples:
  \[
    \forall i < N \st
      Q_1(i)*Q_2(i) \proves \WP{\m[\I2: (t_1'\p;t_2')]}{Q_1(i+1)*Q_2(i+1)}
  \]
  By \ref{rule:wp-seq} and \ref{rule:wp-frame} we can reduce the goal to
  \[
    Q_1(i)*Q_2(i) \proves
    \WP{\m[\I2: t_1']}{Q_1(i+1)} *
    \WP{\m[\I2: t_2']}{Q_2(i+1)}
  \]
  which we can prove by showing the two triples:
  \begin{align*}
    Q_1(i) &\proves
    \WP{\m[\I2: t_1']}{Q_1(i+1)}
    &
    Q_2(i) &\proves
    \WP{\m[\I2: t_2']}{Q_2(i+1)}
  \end{align*}
  We focus on the former as the latter can be dealt with symmetrically.
  By unfolding $Q_1$ we obtain:
  \[
    \WP{\m[\I1: \Loop{i}{t_1}]}{P_1(i)}
    \proves
    \WP{\m[\I2: t_1']}[\big]{\WP{\m[\I1: \Loop{(i+1)}{t_1}]}{P_1(i+1)}}.
  \]
  We then apply \ref{rule:wp-loop-unf} to the innermost WP and \ref{rule:wp-nest}to swap the two WPs in the conclusion:
  \[
    \WP{\m[\I1: \Loop{i}{t_1}]}{P_1(i)}
    \proves
    \WP{\m[\I1: \Loop{i}{t_1}]}[\big]{\WP{\m[\I2: t_1', \I1: t_1]}{P_1(i+1)}}.
  \]
  Finally, by \ref{rule:wp-cons} we can eliminate the topmost WPs
  on both sides and reduce the goal to assumption~\eqref{wp-loop-mix:P1}.
\end{proof}
