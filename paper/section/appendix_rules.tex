\section{The Rules of \thelogic}
\label{sec:appendix:rules}

In this section we list all the rules of \thelogic,
including some omitted (but useful) rules in addition to those
that appear in the main text.
Since our treatment is purely semantic,
rules are simply lemmas that hold in the model.
Although we do not aim for a full axiomatization,
we try to identify the key proof principles that apply to each of our connectives.
For brevity, we omit the rules that apply to the basic connectives of separation logic, as they are well-known and have been proven correct for any model that is an RA. For those we refer to \cite{KrebbersJ0TKTCD18}.

In \cref{fig:assertions} we summarize the notation we use for
assertions over \thelogic's model.
Recall that \thelogic's assertions
$P \in \HAssrt_I \is \Model_I \ucto \Prop $
are the upward-closed predicates over elements of
the RA $\Model_I$.

\begin{figure}[h]
\adjustfigure \begin{align*}
    \pure{\varphi} &\is \fun \wtv. \varphi
    \\
    \Own{b} &\is \fun a. b \raLeq a
    \\
    P \land Q &\is \fun a.
        P(a) \land Q(a)
    \\
    P * Q &\is \fun a.
      \exists b_1,b_2 \st
        (b_1 \raOp b_2) \raLeq a \land
        P(b_1) \land
        Q(b_2)
    \\
    \E x \of X. K &\is \fun a.
      \exists x \of X \st
        K(x)(a)
    & (K\from X \to \HAssrt_I)
    \\
    \A x \of X. K &\is \fun a.
      \forall x \of X \st
        K(x)(a)
    & (K\from X \to \HAssrt_I)
    \\
    \Own{\m{\salg}, \m{\prob}} &\is
      \E \m{\permap}. \Own{\m{\salg}, \m{\prob}, \m{\permap}}
    \\
    \distAs{\aexpr\at{i}}{\prob} & \is
      \E \m{\salg},\m{\prob}.
      \Own{\m{\salg},\m{\prob}} *
      \pure{
        \almostM{\aexpr}{(\m{\salg}(i),\m{\prob}(i))}
        \land
        \prob = \m{\prob}(i) \circ \inv{\aexpr}
      }
    \\
    \CMod{\prob} K &\is
    \fun a.
      \begin{array}[t]{@{}r@{\,}l@{}}
        \E \m{\sigmaF}, \m{\mu}, \m{\permap}, \m{\krnl}.
        & (\m{\sigmaF}, \m{\mu}, \m{\permap}) \raLeq a
        \land
        \forall i\in I\st
        \m{\mu}(i) = \bind(\prob, \m{\krnl}(i))
       \\ & \land \;
        \forall v \in \psupp(\prob).
          K(v)(\m{\sigmaF}, \m{\krnl}(I)(v), \m{\permap})
      \end{array}
    & (\prob \of \Dist(A), K\from A \to \HAssrt_I)
    \\
    \WP {\m{t}} {Q} &\is
      \fun a.
        \forall \m{\prob}_0.
          \forall c \st
          (a \raOp c) \raLeq \m{\prob}_0
          \implies
          \exists b \st
          \bigl(
            (b \raOp c) \raLeq \sem{\m{t}}(\m{\prob}_0)
            \land
            Q(b)
          \bigr)
    \\
    \sure{\aexpr\at{i}} &\is
\distAs{(\aexpr \in \true)\at{i}}{\dirac{\True}}
    \\
    \own{\aexpr\at{i}} &\is
      \E \prob. \distAs{\aexpr\at{i}}{\prob}
    \\
    \perm{\ip{x}{i}:q} &\is
      \E\m{\psp},\m{\permap}.
        \Own{\m{\psp},\m{\permap}}
        * \pure{\m{\permap}(i)(\p{x}) = q}
    \\
    P\withp{\m{\permap}} &\is
      P \land \E\m{\psp}.\Own{\m{\psp}, \m{\permap}}
    \\
    \cpl{R} &\is
      \E \prob \of \Dist(\Val^{X}).
        \pure{\prob(R) = 1} *
        \CC\prob \m{v}.
          \sure{\ip{x}{i} = \m{v}(\ip{x}{i})}_{\ip{x}{i}\in X}
    & (R \subs \Val^{X}, X \subs I \times \Var)
  \end{align*}
\caption{The assertions used in \thelogic.}
\label{fig:assertions}
\end{figure}

\begin{proposition}[Upward-closure]
  All the assertions in \cref{fig:assertions} are upward-closed.
\end{proposition}
\begin{proof}
  Easy by inspection of the definitions.
  The definitions where upward-closedness is less obvious
  (\eg \supercond)
  are made upward-closed by construction by explicit use of
  the order $\raLeq$ in the definition.
\end{proof}

Although we adopt a ``shallow embedding'' approach to assertions
(and thus we do not provide separate syntax),
the rules of \thelogic\ provide an axiomatic treatment
of these assertions so that the user should never manipulate raw predicates
over the semantic model.
We consider the connectives listed above WP
(included) in \cref{fig:assertions} to be the ones that the user
should never need to unfold into their definitions and only manipulate
through rules.

We make a distinction between
``primitive'' and ``derived'' rules.
The primitive rules require proofs that manipulate the semantic model definitions directly; these are the ones we would consider part of a proper axiomatization.
The derived rules can be proved sound by staying at the level of the logic,
\ie by using the primitive rules of \thelogic.

\Cref{fig:primitive-rules} lists the primitive rules for distribution ownership assertions and for the \supercond\ modality.
\Cref{fig:wp-rules} lists the primitive rules for the weakest precondition modality.
In \cref{fig:derived-rules} we list some useful derived rules,
and in \cref{fig:derived-wp-rules} we show some derived rules for WP.

We provide proofs for each rule in the form of lemmas in \cref{sec:appendix:soundness}.
The name labelling each rule is a link to the proof of soundness of the rule.

\let\RuleName\RuleNameProofLink

\begin{figure}[btp]
\adjustfigure[\small]
  \rulesection*{Distribution ownership rules}
  \begin{proofrules}
    \infer*[lab=and-to-star]{
  \idx(P) \inters \idx(Q) = \emptyset
}{
  P \land Q \proves P \sepand Q
}     \relabel{rule:and-to-star}

    \infer*[lab=dist-inj]{}{
  \distAs{\aexpr\at{i}}{\prob}
  \land
  \distAs{\aexpr\at{i}}{\prob'}
  \proves
  \pure{\prob=\prob'}
}     \label{rule:dist-inj}

    \infer*[lab=sure-merge]{}{
  \sure{\aexpr_1\at{i}} *
  \sure{\aexpr_2\at{i}}
  \lequiv
  \sure{(\aexpr_1 \land \aexpr_2)\at{i}}
}
%
     \label{rule:sure-merge}

    \infer*[lab=sure-and-star]{
  \psinv(P, \pvar(E\at{i}))
}{
  \sure{E\at{i}} \land P
  \proves
  \sure{E\at{i}} \ast P
}
%
     \label{rule:sure-and-star}

    \infer*[lab=prod-split]{}{
  \distAs{(\aexpr_1\at{i}, \aexpr_2\at{i})}{\prob_1 \otimes \prob_2}
  \proves
  \distAs{\aexpr_1\at{i}}{\prob_1} *
  \distAs{\aexpr_2\at{i}}{\prob_2}
}
     \relabel{rule:prod-split}
  \end{proofrules}
\rulesection{\Supercond rules}
\begin{proofrules}
    \infer*[lab=c-true]{}{
  \proves \CC\prob \wtv.\True
}     \relabel{rule:c-true}

    \infer*[lab=c-false]{}{
  \CC\prob v.\False
  \proves
  \False
}     \label{rule:c-false}

    \infer*[lab=c-cons]{
  \forall v\st
  K_1(v) \proves K_2(v)
}{
  \CC\prob v.K_1(v)
  \proves
  \CC\prob v.K_2(v)
}     \relabel{rule:c-cons}

    \infer*[lab=c-frame]{}{
  P * \CC\prob v.K(v)
  \proves
  \CC\prob v.(P * K(v))
}     \relabel{rule:c-frame}

    \infer*[lab=c-unit-l]{}{
  \CC{\dirac{v_0}} v.K(v)
  \lequiv
  K(v_0)
}     \relabel{rule:c-unit-l}

    \infer*[lab=c-unit-r]{}{
  \distAs{\aexpr\at{i}}{\mu}
  \lequiv
  \CC\prob v.\sure{\aexpr\at{i}=v}
}     \relabel{rule:c-unit-r}

    \infer*[lab=c-assoc]{
\prob_0 = \bind(\prob,\fun v.(\bind(\krnl(v), \fun w.\return(v,w))))
}{
  \CC{\prob} v.\CC{\krnl(v)} w.K(v,w)
  \proves
  \CC{\prob_0} (v,w).K(v,w)
}     \label{rule:c-assoc}

    \infer*[lab=c-unassoc]{}{
  \CC{\bind(\prob,\krnl)} w.K(w)
  \proves
  \CC\prob v. \CC{\krnl(v)} w.K(w)
}     \relabel{rule:c-unassoc}

    \infer*[lab=c-and]{
  \idx(K_1) \inters \idx(K_2) = \emptyset
}{
  \CC{\prob} v. K_1(v)
    \land
  \CC{\prob} v. K_2(v)
  \proves
  \CC\prob v.
    (K_1(v) \land K_2(v))
}     \relabel{rule:c-and}

    \infer*[lab=c-skolem]{
  \prob \of \Dist(\Full{A})
}{
  \CC\prob v. \E x \of X. Q(v, x)
  \proves
  \E f \of A \to X. \CC\prob v. Q(v, f(v))
}     \relabel{rule:c-skolem}

    \infer*[lab=c-transf]{
f \from \psupp(\prob') \to \psupp(\prob)
  \;\text{ bijective}
  \\\\
  \forall b \in \psupp(\prob') \st
    \prob'(b) = \prob(f(b))
}{
  \CC\prob a.K(a)
  \proves
  \CC{\prob'} b.K(f(b))
}     \relabel{rule:c-transf}

    \infer*[lab=sure-str-convex]{}{
  \CC\prob v.(K(v) * \sure{\aexpr\at{i}})
  \proves
  \sure{\aexpr\at{i}} * \CC\prob v.K(v)
}     \relabel{rule:sure-str-convex}

    \infer*[lab=c-for-all]{}{
  \CC\prob v. \A x \of X. Q(v, x)
  \proves
  \A x \of X. \CC\prob v. Q(v, x)
}     \label{rule:c-for-all}

    \infer*[lab=c-pure]{}{
  \pure{\prob(\event)=1} * \CC\prob v.K(v)
  \lequiv
  \CC\prob v.(\pure{v \in \event} * K(v))
}     \relabel{rule:c-pure}
  \end{proofrules}
  \rulesectionend
\caption{The primitive rules of \thelogic.}
\label{fig:primitive-rules}
\end{figure}

\begin{figure}[tp]
\adjustfigure[\small]
  \rulesection*{Structural WP rules}
  \begin{proofrules}
    \infer*[lab=wp-cons]{
  Q \proves Q'
}{
  \wpc{\m{t}}{Q}
  \proves
  \wpc{\m{t}}{Q'}
}     \relabel{rule:wp-cons}

    \infer*[lab=wp-frame]{}{
  P \sepand \wpc{\hpt}{Q}
  \proves
  \wpc{\hpt}{\liftA{P} \sepand Q}
}     \relabel{rule:wp-frame}

    \infer*[lab=wp-nest]{}{
  \wpc{\m{t}_1}{
    \wpc{\m{t}_2}{Q}
  }
  \lequiv
  \wpc{(\m{t}_1 \m. \m{t}_2)}{Q}
}     \relabel{rule:wp-nest}

    \infer*[lab=wp-conj]{
  \idx(Q_1) \inters \supp{\m{t}_2} \subs \supp{\m{t}_1}
    \\
  \idx(Q_2) \inters \supp{\m{t}_1} \subs \supp{\m{t}_2}
}{
  \wpc{\m{t}_1}{Q_1}
  \land
  \wpc{\m{t}_2}{Q_2}
  \proves
  \wpc{(\m{t}_1 \m+ \m{t}_2)}{Q_1 \land Q_2}
}     \label{rule:wp-conj}

    \infer*[lab=c-wp-swap]{}{
  \CC\prob v.
    \WP{\m{t}}{Q(v)}
    \land \ownall
  \proves
  \WP{\m{t}}*{\CC\prob v.Q(v)}
}     \relabel{rule:c-wp-swap}
  \end{proofrules}
  \rulesection{Program WP rules}
  \begin{proofrules}
    \infer*[lab=wp-skip]{}{
  P \proves \WP {\m[i: \code{skip}]} {P}
}     \label{rule:wp-skip}

    \infer*[lab=wp-seq]{}{
  \WP {\m[i: t]}[\big]{
    \WP {\m*[i: \smash{t'}]} {Q}
  }
  \proves
  \WP {\m[i: (t\code{;}\ t')]} {Q}
}     \relabel{rule:wp-seq}

    \infer*[lab=wp-assign]{
  \p{x} \notin \pvar(\expr)
  \\
  \forall \p{y} \in \pvar(\expr) \st
    \m{\permap}(\ip{y}{i}) > 0
  \\
  \m{\permap}(\ip{x}{i})=1
}{
  (\m{\permap})
  \proves
  \WP {\m[i: \code{x:=}\expr]}[\big] {
    \sure{\p{x}\at{i} = \expr\at{i}}\withp{\m{\permap}}
  }
}
     \relabel{rule:wp-assign}

    \infer*[lab=wp-samp]{}{
  \perm{\ip{x}{i} : 1}
  \proves
  \WP {\m[i: \code{x:~$\dist$($\vec{v}$)}]}
      {\distAs{\ip{x}{i}}{\dist(\vec{v})}}
}     \relabel{rule:wp-samp}

    \infer*[lab=wp-if-prim]{}{
{\begin{array}{@{}r@{}l@{}}
  &\ITE{v}
    {\WP{\m[i: t_1]}{Q(1)}}
    {\WP{\m[i: t_2]}{Q(0)}}
\\\proves{}&
  \WP{
    \m[i: \code{if $\;v\;$ then $\;t_1\;$ else $\;t_2$}]
  }{Q(v)}
\end{array}}
}     \label{rule:wp-if-prim}

    \infer*[lab=wp-bind]{}{
  \sure{\expr\at{i}=v} *
    \WP{\m*[i: {\Ectxt[v]}]}{Q}
  \proves
  \WP{\m*[i: {\Ectxt[\expr]}]}{Q}
}     \label{rule:wp-bind}

    \infer*[lab=wp-loop-unf]{}{
  {\begin{array}{@{}r@{}l@{}}
    &\WP {\m[i: \Loop{n}{t}]} {
      \WP {\m[i: t]} { Q }
    }
    \\\proves{}&
    \WP {\m[i: \Loop{(n+1)}{t}]} {Q}
  \end{array}}
}     \relabel{rule:wp-loop-unf}

    \infer*[lab=wp-loop,right=$n\in\Nat$]{
  \forall i < n\st
  P(i) \proves
  \WP {\m[j: t]} {P(i+1)}
}{
  P(0) \proves
  \WP {\m[j: \Loop{n}{t}]} {P(n)}
}     \relabel{rule:wp-loop}
  \end{proofrules}
  \rulesectionend
\caption{The primitive WP rules of \thelogic.}
\label{fig:wp-rules}
\end{figure}


\begin{figure}[tp]
\adjustfigure[\small]
  \rulesection*{Ownership and distributions}
  \begin{proofrules}
    \infer*[lab=sure-dirac]{}{
  \distAs{\aexpr\at{i}}{\dirac{v}}
  \lequiv
  \sure{\aexpr\at{i}=v}
}     \label{rule:sure-dirac}

    \infer*[lab=sure-eq-inj]{}{
  \sure{\aexpr\at{i} = v}
  *
  \sure{\aexpr\at{i} = v'}
  \proves
  \pure{v=v'}
}
%
     \label{rule:sure-eq-inj}

    \infer*[lab=sure-sub]{}{
  \distAs{\aexpr_1\at{i}}{\prob}
  *
  \sure{(\aexpr_2 = f(\aexpr_1))\at{i}}
  \proves
  \distAs{\aexpr_2\at{i}}{\prob \circ \inv{f}}
}
%
     \label{rule:sure-sub}

    \infer*[lab=dist-fun]{}{
  \distAs{\aexpr\at{i}}{\prob}
  \proves
  \distAs{(f\circ\aexpr)\at{i}}{\prob \circ \inv{f}}
}
%
     \label{rule:dist-fun}

    \infer*[lab=dirac-dup]{}{
  \distAs{\aexpr\at{i}}{\dirac{v}}
  \proves
  \distAs{\aexpr\at{i}}{\dirac{v}} *
  \distAs{\aexpr\at{i}}{\dirac{v}}
}     \label{rule:dirac-dup}

    \infer*[lab=dist-supp]{}{
  \distAs{\aexpr\at{i}}{\prob}
  \proves
  \distAs{\aexpr\at{i}}{\prob} * \sure{\aexpr\at{i} \in \psupp(\prob)}
}
%
     \label{rule:dist-supp}

    \infer*[lab=prod-unsplit]{}{
  \distAs{\aexpr_1\at{i}}{\prob_1} *
  \distAs{\aexpr_2\at{i}}{\prob_2}
  \proves
  \distAs{(\aexpr_1\at{i}, \aexpr_2\at{i})}{\prob_1 \otimes \prob_2}
}
%
     \label{rule:prod-unsplit}
    \end{proofrules}
    \rulesection{\Supercond}
    \begin{proofrules}
      \infer*[lab=c-fuse]{}{
  \CC{\prob} v.
  \CC{\krnl(v)} w.
    K(v,w)
  \lequiv
  \CC{\prob \fuse \krnl} (v,w). K(v,w)
}       \relabel{rule:c-fuse}

      \infer*[lab=c-swap]{}{
  \CC{\prob_1} v_1.
    \CC{\prob_2} v_2.
      K(v_1, v_2)
  \proves
  \CC{\prob_2} v_2.
    \CC{\prob_1} v_1.
      K(v_1, v_2)
}       \label{rule:c-swap}

      \infer*[lab=sure-convex]{}{
  \CC\prob v.\sure{\aexpr\at{i}}
  \proves
  \sure{\aexpr\at{i}}
}       \label{rule:sure-convex}

      \infer*[lab=dist-convex]{}{
  \CC\prob v.\distAs{\aexpr\at{i}}{\prob'}
  \proves
  \distAs{\aexpr\at{i}}{\prob'}
}       \label{rule:dist-convex}

      \infer*[lab=c-sure-proj]{}{
  \CC\prob (v, w).\sure{\aexpr(v)\at{i}}
  \lequiv
  \CC{\prob\circ\inv{\proj_1}} v.\sure{\aexpr(v)\at{i}}
}       \label{rule:c-sure-proj}

      \infer*[lab=c-sure-proj-many]{}{
  \CC\prob (\m{v}, w).
    \sure{\ip{x}{i}=\m{v}(\ip{x}{i})}_{\ip{x}{i}\in X}
  \lequiv
  \CC{\prob\circ\inv{\proj_1}} \m{v}.
    \sure{\ip{x}{i}=\m{v}(\ip{x}{i})}_{\ip{x}{i}\in X}
}       \label{rule:c-sure-proj-many}

      \infer*[lab=c-extract]{}{
  \CC{\prob_1} v_1. \bigl(
    \sure{\aexpr_1\at{i} = v_1} *
    \distAs{\aexpr_2\at{i}}{\prob_2}
  \bigr)
  \proves
  \distAs{\aexpr_1\at{i}}{\prob_1} *
  \distAs{\aexpr_2\at{i}}{\prob_2}
}       \relabel{rule:c-extract}

      \infer*[lab=c-dist-proj]{}{
  \CC\prob (x, y).
    \distAs{\aexpr\at{i}(x)}{\prob(x)}
  \proves \CC{\prob\circ\inv{\proj_1}} x.
    \distAs{\aexpr\at{i}(x)}{\prob(x)}
}       \label{rule:c-dist-proj}
    \end{proofrules}
    \rulesection{Relational Lifting}
    \begin{proofrules}
    \infer*[lab=rl-cons]{
  R_1 \subs R_2
}{
  \cpl{R_1} \proves \cpl{R_2}
}
%
     \label{rule:rl-cons}

    \infer*[lab=rl-unary]{
  R \subs \Val^{\set{\p{x}_1\at{i},\dots,\p{x}_n\at{i}}}
}{
  \cpl{R} \proves \sure{R(\p{x}_1\at{i},\dots,\p{x}_n\at{i})}
}
%
     \label{rule:rl-unary}

    \infer*[lab=rl-eq-dist]{i \ne j}{
  \cpl{\ip{x}{i}=\ip{y}{j}}
  \proves
  \E \prob.
    \distAs{\ip{x}{i}}{\prob}
    *
    \distAs{\ip{y}{j}}{\prob}
}     \label{rule:rl-eq-dist}

    \infer*[lab=rl-convex]{}{
  \CC\prob \wtv.\cpl{R} \proves \cpl{R}
}
%
     \relabel{rule:rl-convex}

    \infer*[lab=rl-merge]{}{
  \cpl{R_1} * \cpl{R_2}
  \proves
  \cpl{R_1 \land R_2}
}
%
     \relabel{rule:rl-merge}

    \infer*[lab=rl-sure-merge]{
  R \subs \Val^{X}
  \\
  \pvar(\expr\at{i}) \subs X
}{
  \cpl{R} * \sure{\ip{x}{i} = \expr\at{i}}
  \proves
  \cpl{R \land \p{x}\at{i} = \expr\at{i}}
}     \label{rule:rl-sure-merge}

    \infer*[lab=coupling]{
  \prob \circ \inv{\proj_1} = \prob_1
  \\
  \prob \circ \inv{\proj_2} = \prob_2
  \\
  \prob(R) = 1
}{
  \distAs{\p{x}_1\at{\I1}}{\prob_1} *
  \distAs{\p{x}_2\at{\I2}}{\prob_2}
  \proves
  \cpl{R(\p{x}_1\at{\I1}, \p{x}_2\at{\I2})}
}
     \relabel{rule:coupling}\end{proofrules}
  \rulesectionend
\caption{Derived rules.}
\label{fig:derived-rules}
\end{figure}

\begin{figure}[tp]
\adjustfigure[\small]
\begin{proofrules}
    \infer*[lab=wp-loop-0]{}{
  P \proves \WP {\m[i: \Loop{0}{t}]} {P}
}     \label{rule:wp-loop-0}

    \infer*[lab=wp-loop-lockstep,right=$n\in\Nat$]{
  \forall k < n\st
    P(k) \proves \WP {\m[i: t, j: t']}{P(k+1)}
}{
  P(0) \proves
  \WP {\m[i: (\Loop{n}{t}), j: (\Loop{n}{t'})]} {P(n)}
}     \label{rule:wp-loop-lockstep}

    \infer*[lab=wp-rl-assign]{
  R \subs \Val^{X}
  \\
  \ip{x}{i} \notin \pvar(\expr\at{i}) \subs X
  \\
  \forall \p{y} \in \pvar(\expr) \st
    \m{\permap}(\ip{y}{i}) > 0
  \\
  \m{\permap}(\ip{x}{i})=1
}{
  \cpl{R}\withp{\m{\permap}}
  \proves
  \WP {\m[i: \code{x:=}\expr]}[\big] {
    \cpl{R \land \p{x}\at{i} = \expr\at{i}}\withp{\m{\permap}}
  }
}     \label{rule:wp-rl-assign}

    \infer*[lab=wp-if-unary]{
  P * \sure{\ip{e}{\I1} = 1}
  \gproves
  \WP {\m[\I1: t_1]}{Q(1)}
  \\
  P * \sure{\ip{e}{\I1} = 0}
  \gproves
  \WP {\m[\I1: t_2]}{Q(0)}
}{
  P * \distAs{\ip{e}{\I1}}{\beta}
  \gproves
  \WP {\m[\I1: \Cond{\p{e}}{t_1}{t_2}]}[\big]{\CC{\beta} b.Q(b)}
}     \label{rule:wp-if-unary}
  \end{proofrules}
\caption{Derived WP rules.}
\label{fig:derived-wp-rules}
\end{figure}

